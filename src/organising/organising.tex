Different methods of software development exists in software engineering. A different type of development means a different way of testing, so it is important to choose a proper development process before a team starts developing. This chapter expains the development methods and how to organise performance regression testing when using each of these methods.

\section{Code-and-fix model}
``The basic model used during the early days of software devolpment contained two steps: write some code and fix the problems in the code'' \cite{boehm1988spiral} This `method' is not appropriate for performance regression testing, because of the poor preparation of testing. Without any idea how the overall design will be, will make the writing of test suites even harder.

\section{Waterfall method}
``The waterfall method establishes a sequence of stages-requirements, specifications, design, coding, testing and maintenance-to guide the development process. ''\cite{kang1989software}. The sequences of the waterfall method will be done until the development of the system is completed. The waterfall method even became the basis for most software acquisition standards in the past years. \cite{boehm1988spiral}. This type of developing has a testing phase. Collins et al researched an agile waterfall process. \cite{collins2010iterative} The research showed that, when using an iterative process, regression testing using the waterfall method can be very effective. This implies direclty that performance regression testing will be effective here, because most of the time performance is the biggest issue in the field. \cite{foo2010mining}

\section{Spiral method}
``The spiral method creates a risk-driven approach to the software process rather than a primarily document-driven or code-driven process. It incorporates many of the strengths
of other models and resolves many of their
difficulties.''\cite{boehm1988spiral} When using the spiral model, the main focus lies on the different risks of a software project. Examples of these risks are: low budget, developing wrong functions or a continous change of functions. These development processes are divided by the property of their risks. So performance regression testing would not be appropriate for the spiral method, because performance regression tests are not seen as risks.

\section{Scrum}
``Scrum is an Agile software development process designed to add energy, focus, clarity, and transparency to project teams developing software systems.''\cite{sutherland2007distributed} A scrum development process is divided into three kinds of phases. Research has been done to show that automated regression tests during the daily builds will greatly improve the quality of the product \cite{Future_of_Scrum}. This means that scrum is a useful development method for performance regression testing. \\ There are two organising phases, which are the opening and closing phases and the last phase is the sprint phase which is further divided. The planning and closing phases are organising phases. It is not necessary to run tests here, because there is either no code (opening) or all the performance regression tests have been executed and passed (closing). Though, planning when to test is an example of an organising aspect which can be done during the opening phase and verifying that the test suite covers all the code can be done during the closing phase. \\ The actual creating of tests will be done during the sprints. The sprints consist of developing, wrapping, reviewing and adjusing. The wrapping part of the sprint, combines all the implemented code. This process of wrapping is a very appropriate moment to test for performance regressions. \\

\section{General aspects of test origanization}
Despite the fact that the development method is significant, there are some other organising aspects which could make a difference in the way of performance regression testing. \\
First of all, it is important who will do the testing. Zaparanuks et al. performed a research where one person did all the testing throughout the project. \cite{sutherland2009fully} This person is an active member of the team and watches all the members of the developing team. This method provides a way to deal with any issues that could come up during the process. The appointed person will directly try to deal with issues coming up by the members of the developing team. By appointing this person, the quality of the product is watched the whole time during the project, instead of comparison to the other research where performance regression testing is done during the merge phase. This research proved that the quality of the development got a much higher average score compared to other similar development products. So appointing one tester who only has the task to watch over the quality by testing and for our sake performing regression tests, has an improved effect on the overall process. \\

Another organising aspect is to make sure a proper test suite is written before programming, and that this test suite is maintained during the project. Test suites should be made with some conditions in mind. Writing complete test suites for complex systems will require the regression testing to run for a long time. \cite{rothermel2001prioritizing} So the developers have to decide to either test the complete system, or to write selective test suites. These test suites will only test the main functionality of the system. \\
After deciding the organising aspects, the actual implementing will be initiated. From now on performance regression testing will be done throughout the process. The developers will decide what kind of data of the implementation will be tested. Performance counters can be used to help decide which data the tests will cover.
