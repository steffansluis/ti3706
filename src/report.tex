\documentclass[oneside]{book}

\usepackage{fullpage,hyperref,titlesec,enumerate,babelbib,graphicx,float}
\usepackage[english]{babel}
\usepackage{titlesec} % required for removing header spacing

% remove header spacing
\titlespacing\section{0pt}{12pt plus 4pt minus 2pt}{0pt plus 2pt minus 2pt}
\titlespacing\subsection{0pt}{12pt plus 4pt minus 2pt}{0pt plus 2pt minus 2pt}
\titlespacing\subsubsection{0pt}{12pt plus 4pt minus 2pt}{0pt plus 2pt minus 2pt}

% remove all indents
\setlength{\parindent}{0pt}

% Add all chapter folders to graphics path
\graphicspath{
% {./introduction/}
}

\titleformat{\chapter}[hang]{\Huge\bfseries}{\thechapter. }{0pt}{\Huge\bfseries}

\begin{document}
\frontmatter
%%% TITELPAGE %%%

\begin{titlepage}

\begin{center}
\begin{figure}[h!]
\centering
% Title image
% \includegraphics[scale=1]{Images/TU_P2_black.png}
\end{figure}
Faculty of Computer Science \\
\today

\vspace{3.5cm}
\selectfont
{\Large TN3706 - Bachelor Seminar}\\
\vspace{0.0cm}
\Huge{\textbf{Performance Regression Testing\\ A developer's perspective}}
\vspace{0.5cm}
\selectfont

\vspace{5cm}
\normalsize{\textbf{Bachelor Seminar Group}}

\begin{tabular}{ l r}
\normalsize{M.A. Hoppenbrouwer} & \normalsize{4243889} \\
\normalsize{M.J. Otte} & \normalsize{4222695} \\
\normalsize{R.S. Sluis} & \normalsize{4088816} \\
\normalsize{R. Vink} & \normalsize{4233867}\\
\end{tabular}

\vspace{0.75cm}

\begin{tabular}{ l l }
\normalsize{M. Loog} & \normalsize{Professor} \\
\normalsize{C. Bezemer} & \normalsize{Supervisor} \\
\end{tabular}

\end{center}
\end{titlepage}

\chapter{Preface}
% In the growing world of tools and methods that are available to a software developer, it becomes harder and harder to keep track of which technique is best to use under what circumstances. This holds true for the growing field of performance regression testing, where new testing strategies are being discovered and discussed, and the steps in the process of detecting performance regressions are being improved constantly. This literature study aims to give an overview of these steps in a way that links each step to a point in the software development process corresponding to the challenges that arise when dealing with performance regression testing in practice.


\chapter{Summary}
% Performance regression testing brings challenges with it. In this paper we discuss what these challenges are, how they are handled and what problems still remain. It shows when it is useful to start performance regression testing during the development. How the data for the tests are gathered. What can be done to process the gathered data. And how to report on it. Ultimately there is still work to be done in the subject, but the paper shows that performance regression testing is a useful method of testing.


\clearpage % To point the bookmark to the top of the page
\pdfbookmark{\contentsname}{toc}
\tableofcontents

\clearpage % To point the bookmark to the top of the page
\pdfbookmark{\listfigurename}{lof}
\listoffigures

\clearpage % To point the bookmark to the top of the page
\pdfbookmark{\listtablename}{lot}
\listoftables

\mainmatter
%  INTRODUCTION
% The definition of performance regression testing is: ``Performance regression testing detects performance
regressions in a system under load. Such regressions refer to
situations where software performance degrades compared to
previous releases, although the new version behaves correctly.''\cite{foo2010mining}
Performance regression testing can be seen as the combination of regression testing and performance testing. ``Regression tesing is the retesting of software following modifications.''\cite{rothermel2001prioritizing} Performance testing is testing performance requirements and specifications of it.\cite{gan2006software}

To improve the quality of the software product it could be useful to use performance regression testing in a way the developers can
directly improve the quality of the software product. This will be the main focus of this article. There are different aspects in optimizing performance regression tests, which will be discussed throughout this paper. Every aspect will be clarified in the form of a new chapter in this paper. \\ First of all it needs to be clear how to acquire data and what kind of performance counters will be used for performance regression testing. Next is the filtering of the performance metrics, used to obtain differences in data. The following chapter discusses how to compare the filtered data. Last aspect is that the filtered performance metrics can be reported in the form of visualizations so performance regressions will be detected. At the end of this paper, a research agenda and a conclusion will be provided.



\chapter{Chapter 1}
\label{chapter:first}

This chapter discusses how perfomance counters can give information about the quality of software. Performance counters are a type of data output as a result of performance regression testing. This can be e.g. CPU usage, memory usage or disk IO time.

To detect if performance regression occurs, the results of the previous and the current performance regression test need to be compared.




% CONCLUSION
\newpage
% This paper shows that performance regression testing could help the developer improve their system. To do so the developer needs to gather performance metrics, process this data and draw meaningful conclusions from it. If all these steps are made, the developer can tell what kind of performance regressions have occurred, so they can adjust their system. Performance regression testing is a rather new subject in the field Computer Science. This means that it can be further investigated. Some of these future experiments can be found in the research agenda.


% Appendices
\appendix


% END appendices

\backmatter

\phantomsection
\addcontentsline{toc}{chapter}{\bibname}

\bibliographystyle{unsrt}
\bibliography{bibliography}

\end{document}
