After acquiring, filtering and comparing the data, it has to be reported. The following chapter will show the reporting of the results of performance regression testing. It will touch on visualization and dectection of the performance regressions. \newline
\newline
As mentioned in chapter 2, the choice of performance metric is very important. When making a report, this is what it will be based around. A report based on cpu time will have a different structure than a report based on bandwith. The performance metrics require additional data to be able to show regressions. This additional data can be number of times the performance regression test has been executed or the number of times a particular function has been called. This kind of data can be usefull to determine if a performance regression has occured. If due to a new functionality a function is expected to be called twice as much since the last revision, a increase in performance can be expected.\newline

\section{Visualization}
When a report is created, a large amount of data can be usefull. Only it is very hard and error sensitive to look over those numbers. Visualization can be used to show the data in a way that might be easier to comprehend. By using visual representations such as a graph, can be used to spot a change in revisions.

\section{detection of performance regressions}





%mogelijk report techniek: transaction profile
%http://onlinelibrary.wiley.com/doi/10.1002/stvr.1573/pdf
%average precision and recall
%http://sail.cs.queensu.ca/publications/pubs/qsic2010_foo.pdf
