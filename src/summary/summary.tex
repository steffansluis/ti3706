Performance regression testing brings challenges with it. This literature study discusses what the challenges are and how they relate to the development process of a typical software developer.
First, it touches on the differences in organising the testing process for different software engineering approaches, along with some general aspects of test organisation. The viability of performance regression testing is closely related to what development method is used.
Then, the different types of test results are explained. Measuring events that relate to the entire system reduces the accurary of the performance representation that can be obtained from these results. The duration of tests and the amount of measurements are also relevant to this accuracy.
After the test results are obtained, they can be processed using different techniques which are described in the next section of this literature study. Each technique has its own type of result, and some techniques might detect performance regressions where others might not. In addition to this, some techniques require manual analysis of the results before they can be used to detect performance regressions, where others can do so automatically.
Most types of results can be visualized in some way. Doing so when reporting may provide a more clear picture of the obtained results and help in detecting performance regressions. For this reason, different ways of visualisation are mentioned as well. In conclusion, an overview of the state of research is given.
