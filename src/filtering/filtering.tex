\paragraph{Control Charts}
Detecting if performance regression has occured manually costs a lot of time. A possible way to analyze the results faster is to use a control chart.
``The goal of control charts is to automatically determine if a deviation in a process is due to common causes, e.g., input fluctuation, or due to special causes, e.g., defects.'' \cite{nguyen2012using} A control chart consists of a baseline data set and a target data set. A baseline data set contains the results of the prior test. Based on this data set an upper and lower limit will be decided. The target data set contains the results of the new test. From these two data sets the violation ratio can be calculated. This is the percentage of the amount of targets outside the upper and lower limit. If performance regression occurs, the violation ratio is too high. It now seems easy to detect if performance regression occurs: a certain threshold on the maximum allowed violation ratio can be chosen, and if the violation ratio is higher than this threshold performance regression has occurred. Unfortunately, this is not the case. There are some reasons that make it hard to detect performance regression. If the violation ratio is low, the probability that performance regression has occurred is low as well. A high violation ratio doesn't always mean that performance regression has occurred. Some performance counters are inconsistent, and show different results on the same input data. Because of this, it is possible that the violation ratio is higher than the chosen threshold but no performance regression has occurred. To avoid this, the prior test will be executed again after the new test.






