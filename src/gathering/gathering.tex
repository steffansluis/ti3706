This chapter will explain how the gathering of useful data will be done. First it will expalin the organising aspects, where the second part is about how to present data in the form of performance counters.
\section{Method of development}
Software development processes have different methods of software development. Different types of development mean different types of testing, so it is important to choose a proper development process before a team starts implementing. This section show different types of development and how to do performance regression testing on those methods.
\subsection{Code-and-fix model}
``The basic model used during the early days of software devolpment contained two steps: write some code and fix the problems in the code'' \cite{boehm1988spiral} This `method' is not appropriate for performance regression testing, because of the poor preparation of testing. 

\subsection{Waterfall method}
``The waterfall method establishes a sequence of stages-requirements, specifications, design, coding, testing and maintenance-to guide the development process. ''\cite{kang1989software}. This method of development and even became the basis for most software acquisition standards. \cite{boehm1988spiral}. The waterfall method can be a good method to test performance regressions. Collins et al researched an iterative waterfall process. \cite{collins2010iterative} The research showed that, when using an iterative process, regression testing using the waterfall method can be very effective. 
\subsection{Spiral metod}
``The spiral method creates a risk-driven approach to the software process rather than a primarily document-driven or code-driven process. It incorporates many of the strengths
of other models and resolves many of their
difficulties.''\cite{boehm1988spiral} When using the spiral model, the main focus is the different risks of a software project. Examples of these risks are: low budget, developing wrong functions or a continous change of functions. These development processes are divided by the property of these risks. So performance regression testing would not be a appropriate for the spiral method, because performance regression tests are not a risk fact.  
\subsection{Scrum} 
``Scrum is an Agile software development process designed to add energy, focus, clarity, and transparency to project teams developing software systems.''\cite{sutherland2007distributed} A scrum development process is divided into three kind of phases. Research has been done to show that automated regression tests during the daily builds will greatly improve the quality of the product \cite{Future_of_Scrum}. This means that scrum is a useful development method for performance regression testing. \\ There are two organising phases, which are the opening and closing phases and the last phase is the sprint phase which is further divided. The planning and closing phases are organising phases. It is not necessary to run tests here, because there is either no code (opening) or all the performance regression tests have been runned and approved (closing). Though, Planning the test suite is an example of an organising aspect which can be done during the opening phase and verifying that the test suite contains all the tests can be done during the closing phase. \\ The actual implementing will be done during the sprints. The sprints consist of developing, wrapping, reviewing and adjusing. The wrapping part of the sprint, combines all the implemented code. This process of wrapping is a very appropriate way to test for performance regressions.