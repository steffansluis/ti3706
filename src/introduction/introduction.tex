In the growing world of tools and methods that are available to software developers, it becomes harder and harder to keep track of which technique is best to use under what circumstances. This holds true for the growing field of performance regression testing. New testing strategies are being discovered and discussed, and the steps in the process of detecting performance regressions are being improved constantly. This literature study gives an overview of these steps in a way that links each step to a point in the software development process. Each chapter describes a step and the corresponding challenges that arise when dealing with performance regression testing in practice.

In short performance regression testing can be defined as followed: \textit{``Performance regression testing detects performance
regressions in a system under load. Such regressions refer to
situations where software performance degrades compared to
previous releases, although the new version behaves correctly''}\cite{foo2010mining}.
Performance regression testing can be seen as the combination of regression testing and performance testing. \textit{``Regression testing is the retesting of software following modifications''}\cite{rothermel2001prioritizing}. Performance testing is testing performance requirements and specifications of it\cite{gan2006software}. \\

To allow developers to improve the quality of a product during development, performance regression testing can be used. This paper will emphasize how to test performance regressions in the most optimal way. Each chapter of this paper explains a different aspect of performance regression testing. \\ 

Different methods of software development exist in software engineering. A different type of development means a different way of testing, so it is important to choose a proper development process before a team starts developing. The second chapter explains these development methods and how to organise performance regression testing when using each of these methods. Furthermore, it will give general information about organising the testing process. \\

Once the testing process has been organised to facilitate the particular software development method being used, the testing can begin. The next challenge that presents itself is what to test for. In the third chapter it will be clarified what kind of information is relevant to the performance of software and how to handle that information in order to obtain the most accurate representation of the underlying performance. \\

When the test results of performance regression testing are available, they can be processed to filter out insignificant information, detect possible perfomance regressions and find the responsible parts of the software. The fourth chapter explains how such processing is done. 

The fifth chapter shows what can be done to report the results of performance regression testing. It will touch on visualization of the possible performance regressions, and the different methods available to do so. \\

At the end of this paper a research agenda is given. This agenda provides an overview of which problems which are not proven by this paper. These quesions could be analyzed and experimented in a later phase. \\
The paper ends with a conclusion, which summarises the whole paper and gives a general conclusion. 