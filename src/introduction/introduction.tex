The definition of performance regression testing is: ``Performance regression testing detects performance
regressions in a system under load. Such regressions refer to
situations where software performance degrades compared to
previous releases, although the new version behaves correctly.''\cite{foo2010mining}
Performance regression testing can be seen as the combination of regression testing and performance testing. ``Regression tesing is the retesting of software following modifications.''\cite{rothermel2001prioritizing} Performance testing is testing performance requirements and specifications of it.\cite{gan2006software}

To improve the quality of the software product it could be useful to use performance regression testing in a way the developers can
directly improve the quality of the software product. This will be the main focus of this article. There are different aspects in optimizing performance regression tests, which will be discussed throughout this paper. Every aspect will be clarified in the form of a new chapter in this paper. \\ First of all it needs to be clear how to acquire data and what kind of performance counters will be used for performance regression testing. Next is the filtering of the performance metrics, used to obtain differences in data. The following chapter discusses how to compare the filtered data. Last aspect is that the filtered performance metrics can be reported in the form of visualizations so performance regressions will be detected. At the end of this paper, a research agenda and a conclusion will be provided.
