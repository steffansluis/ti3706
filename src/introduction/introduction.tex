In the growing world of tools and methods that are available to a software developer, it becomes harder and harder to keep track of which technique is best to use under what circumstances. This holds true for the growing field of performance regression testing, where new testing strategies are being discovered and discussed, and the steps in the process of detecting performance regressions are being improved constantly. This literature study gives an overview of these steps in a way that links each step to a point in the software development process corresponding to the challenges that arise when dealing with performance regression testing in practice.

In short performance regression testing can be defined as followed: \textit{``Performance regression testing detects performance
regressions in a system under load. Such regressions refer to
situations where software performance degrades compared to
previous releases, although the new version behaves correctly.''\cite{foo2010mining}}
Performance regression testing can be seen as the combination of regression testing and performance testing. \textit{``Regression testing is the retesting of software following modifications.''\cite{rothermel2001prioritizing}} Performance testing is testing performance requirements and specifications of it.\cite{gan2006software} \\

To allow developers to improve the quality of a product during development, performance regression testing can be used. This paper will emphasize how to test performance regressions the most optimal way. Each chapter of this paper explains a different aspect of performance regression testing. \\ 

Different methods of software development exist in software engineering. A different type of development means a different way of testing, so it is important to choose a proper development process before a team starts developing. The second chapter explains these development methods and how to organise performance regression testing when using each of these methods. Furthermore will it explain additional organising aspects. \\

Once the testing process has been organised to facilitate the particular software development method that is being used, the testing can finally begin. The next challenge that presents itself is what to test for. What kind of information is relevant to the performance of software and how to handle that information in order to obtain the most accurate representation of the underlying performance will be clarified in the third chapter. \\

The fourth chapter describes how the results of performance regression testing can be filtered. This means that out of all results only the results of the relevant perfomance counters remain, so that the developers have a better overview of what to change. This can be done manually, but this costs a lot of time. In this chapter two different techniques are explained: control charts and association rules. \\

Even though filtering test results can greatly reduce the amount of data that is involved in analyzing testing results, in order to detect performance regressions the results still have to be compared in some way to the results of a previous run of the test suite. The problems that will arise when doing the comparing of performance regressions and some solutions of it will be discussed in the fifth chapter. \\

The sixth chapter shows what can be done to report the results of performance regression testing. After acquiring, filtering and comparing the data, it has to be reported. It will touch on visualization of the possible performance regressions. \\

At the end of this paper a research agenda is given. This agenda provides open questions which are not proven by this paper. These quesions could be analyzed and experimented in a later phase. \\
The paper ends with a conclusion, which summarises the whole paper and gives a general conclusion. 