Spanning the years and the papers that have been discussed in this report, a lot of problems and possible solutions are introduced. This section provides an overview of the problems and solutions per paper, as well as a summary of problems that are yet to be solved.\\

Performance regression testing and load testing share the dependency of results on the system load and the problem of detecting performance problems. Jiang et al. propose an automated approach to detect functional and performance problems that scales well to enterprise systems and provides high precision results \cite{jiang2010automated}. It can be used to analyze the behavior of a system during a load test, and with regards to performance regression testing, to abstract over the load testing data, allowing for load-independent conclusions to be drawn.

Analyzing load tests results to detect performance regressions is
very time consuming due to the large volume of performance
metics. Nguyen et al. propose an approach that uses control charts, a statistical process control technique, to assist performance engineers in identifying test runs with performance regressions. This helps developers in pinpointing the components which cause regressions and determining the causes of regressions in load tests \cite{nguyen2012using}.

Client-server applications are widespread nowadays. Understanding the cascading effects of the various tasks that are sprung by a single request-reply transaction is a challenging task. Kraft et al. address the issue of efficiently diagnosing essential performance changes in application behavior in order to provide timely feedback to application designers and service providers \cite{kraft2009estimating}. They propose a approach based on an application signature built on the concept of transaction latency profiles and transaction signatures. The approach is non-intrusive and based on monotoring data that is typically available in enterprise applications.

When using a transaction profile, the amount of data that populates the database can have an effect on the profile. This data can mainly affect databases, CPU and I/O. \cite{ghaith2015anomaly}

Bezemer et al. use a method for I/O performance counters with a considerable amount of overhead. Thus, the use of this method creates unreliable results for time sensitive metrics\cite{bezemer2014detecting}.

Performance counters can be difficult to configure, may not be programmable or readable from user-level code, and can not discriminate between events caused by different software threads. Software instructions executed to access a counter may affect that same counter. Some factors include the type of infrastructure that is used and the system configuration. However, a lot more factors can be found in further studies \cite{AccuracyPerformanceCounter}.

The identification of performance regressions can be hard \cite{foo2010mining}. Service Level Objectives are used to determine if anomalies have occurred, but these are not used very often. Identifying performance regressions manually can be subjective. Individual analysts can overlook some important performance metrics. Automated identification can be hard as well, because there could be phase shifts in the performance tests. This will make it hard to use classifier techniques.
In addition, the filtering technique used in the paper did not measure some performance regressions. Only the Apriori algorithm was used, where other filtering algorithms might detect other performance regressions.

% If performance regressions are in the in the system from the start of the development, it will not be detected.

