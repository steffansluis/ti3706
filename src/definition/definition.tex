The following chapter will explain the definition of performance regression testing (PRT). PRT is the combination of regression testing with performance testing. First we will explain what regression testing is, then what performance testing is and last we will discuss what the combination of the two means.

\section{Definition of regression testing}
Regression testing tests whether or not the software has new regressions, software bugs, each time a software change or update is made. Regression testing also determines if software changes made in one part of the software affects the other parts. Usually software bugs will occur when the software is updated. That is why regression testing is used a lot nowadays. Regression testing is applied by running the old test scenarios for the changed program. These tests will cover all the functions and input so that every situation will be tested.

\section{Definition of performance testing}
Performance testing is testing in a way so that performance quality won't be lost. Programmers make certain performance demands and test whether or not these demands are fulfilled. For example the time the program takes to log in must be shorter than 1 second. If the software is not able to achieve this demand, the software has to be adjusted.

\section{Definition of PRT}
The combination of performance testing and regression testing is PRT. This means that if software is tested, it will make sure that regressions will be detected and the loss in performance will be registered.

Regression testing focuses on the verifying of correctness of a change \cite{detection_performance_regressions}. Some large researches actually show that performance is most of the time the biggest primary problem in the field \cite{foo2010mining}. \newline To improve the quality of the software product it could be useful to use PRT in a way the developers can directly improve the quality of the software product. A timeline when to make use of PRT in the development process could be useful.
